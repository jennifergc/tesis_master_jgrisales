%\addcontentsline{toc}{chapter}{RESUMEN}
\newpage
\chapter*{Resumen}
\label{sec:resum}
\footnotesize{
\noindent\textbf{TÍTULO:}  ESTUDIO DE LOS EFECTOS DE ACTIVIDAD SOLAR A LARGO PLAZO
SOBRE EL FLUJO DE RAYOS CÓSMICOS SECUNDARIOS EN EL OBSERVATORIO PIERRE AUGER\astfootnote{Trabajo de grado}

\noindent\textbf{AUTORA:} Jennifer Grisales Casadiegos\asttfootnote{Facultad de Ciencias. Escuela de Física. Director: Luis A. Núñez. }

\noindent\textbf{PALABRAS CLAVES: } Rayos Cósmicos Galácticos, Viento Solar, Ciclo Solar, Decrecimientos Forbush, Detectores Chérenkov.

\noindent \textbf{DESCRIPCIÓN: }

En este trabajo de grado, exponemos los resultados de la cuantificación y exploración de los impactos a corto y largo plazo de la actividad solar sobre el flujo de rayos cósmicos secundarios medidos en el Observatorio Pierre Auger, utilizando el \textit{Modo Scaler}. Este sistema de detección de baja energía tiene la capacidad de registrar el flujo de fondo de rayos cósmicos en términos de partículas por segundo. Su principal objetivo es la detección de destellos de rayos gamma y eventos transitorios, como las disminuciones de Forbush. Aunque desde su creación, el Observatorio está optimizado para partículas con energías superiores a $10^{18}eV$ en los últimos años ha surgido un interés renovado en aprovechar las capacidades del arreglo de detectores de superficie para explorar nuevas aplicaciones y áreas de estudio.

%El objetivo principal de esta tesis consiste en analizar la influencia de la actividad solar en la detección de rayos cósmicos galácticos y determinar cómo podemos extraer información pertinente sobre el ciclo solar y el clima espacial a partir de los datos del Observatorio. 
%Para llevar a cabo este propósito, realizamos un análisis exhaustivo desde el año 2006 hasta el 2021, rango de años que contienen el ciclo solar 24 y los contrastamos con datos provenientes de otros experimentos y fuentes, tales como monitores de neutrones y satélites para el caso de los datos de manchas solares y el viento solar. Para ello nos enfocamos en examinar la correlación entre el flujo de rayos cósmicos y los parámetros solares, arrojando una anticorrelación moderada (entre  0.30 y 0.50) para intensidad de rayos cósmicos y el histórico de número de manchas solares tanto para el Observatorio Pierre Auger como para la estación de neutrones Tsumeb que posee una rigidez de corte similar. Posteriormente, mediante un análisis de eventos transitorios determinamos que la mayor incidencia de decrecimientos Forbush ($71.20\%$ de los eventos) con magnitud $>= 2\%$ respecto a todo el periodo de registro del Observatorio, ocurre en la zona de máxima actividad solar, específicamente en la zona que presenta mayor dispersión en el comportamiento de los Scaler medidos. Adicionalmente implementamos un algoritmo manual usando los criterios usados comúnmente para la identificación de Forbush que nos permitió determinar la sensibilidad actual del observatorio en comparación con otros instrumentos. Para tal fin utilizamos el Forbush del 22 de julio del 2015, uno de los más intensos y mayormente estudiados del ciclo solar 24, obteniendo un porcentaje de decrecimiento del $3.01\%$ que es un valor muy por debajo de los que se reportan en la literatura (entre $5.2\%$ y el $15.2\%$) sugiriendo la necesidad de identificar todas las contribuciones del fondo en los datos.


%Los resultados revelan que el Observatorio Pierre Auger exhibe sensibilidad a las variaciones del clima espacial tanto a corto como a largo plazo, y puede utilizarse como un instrumento complementario valioso para la investigación de la actividad solar y sus efectos sobre la Tierra.

Para cumplir con nuestro objetivo, llevamos a cabo un análisis detallado desde 2006 hasta 2021, periodo que abarca el ciclo solar 24. Contrastamos nuestros hallazgos con datos de otras fuentes, incluyendo monitores de neutrones y satélites, que proporcionan información sobre las manchas solares y el viento solar. Nuestro enfoque principal fue examinar la correlación entre el flujo de rayos cósmicos y los parámetros solares. Encontramos una anticorrelación moderada (entre 0.30 y 0.50) entre la intensidad de los rayos cósmicos y el número histórico de manchas solares, tanto para el Observatorio Pierre Auger como para la estación de neutrones Tsumeb, que tiene una rigidez de corte similar.

En un análisis posterior de eventos transitorios, determinamos que la mayoría de los decrecimientos de Forbush (el $71.20\%$ de los eventos) con una magnitud de $2\%$ o superior, ocurren en la zona de máxima actividad solar. Esta zona también presenta la mayor dispersión en las mediciones de los Scaler.
Además, implementamos un algoritmo manual, basado en criterios comúnmente utilizados para la identificación de eventos de Forbush, para determinar la sensibilidad actual del observatorio en comparación con otros instrumentos. Utilizamos el evento de Forbush del 22 de julio de 2015, uno de los más intensos y estudiados del ciclo solar 24, como referencia. Encontramos un porcentaje de decrecimiento del $3.01\%$, significativamente menor que los valores reportados en la literatura (entre $5.2\%$ y $15.2\%$), lo que sugiere la necesidad de identificar todas las contribuciones al fondo en los datos.

}\normalsize
\clearpage