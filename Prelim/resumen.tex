%\addcontentsline{toc}{chapter}{RESUMEN}
\newpage
\chapter*{Resumen}
\label{sec:resum}
\footnotesize{
\noindent\textbf{TÍTULO:}  ESTUDIO DE LOS EFECTOS DE ACTIVIDAD SOLAR A LARGO PLAZO
SOBRE EL FLUJO DE RAYOS CÓSMICOS SECUNDARIOS EN EL OBSERVATORIO PIERRE AUGER\astfootnote{Trabajo de grado}

\noindent\textbf{AUTORA:} Jennifer Grisales Casadiegos\asttfootnote{Facultad de Ciencias. Escuela de Física. Director: Luis A. Núñez. }

\noindent\textbf{PALABRAS CLAVES: } Rayos Cósmicos Galácticos, Viento Solar, Ciclo Solar, Decrecimientos Forbush, Detectores Chérenkov.

\noindent \textbf{DESCRIPCIÓN: }

En este trabajo de grado, se presentan los resultados de un estudio centrado en los impactos a largo plazo de la actividad solar sobre el flujo de rayos cósmicos secundarios en el Observatorio Pierre Auger, utilizando el \textit{Modo Scaler}. Este sistema de detección de baja energía registra el flujo de fondo de rayos cósmicos en términos de partículas por segundo, con el propósito de detectar destellos de rayos gamma y eventos transitorios como las disminuciones de Forbush. A pesar de estar optimizado para partículas con energías superiores a $10^{18}eV$, el Observatorio Pierre Auger implementó este sistema para aprovechar las capacidades del arreglo de detectores Chérenkov.

El objetivo principal de esta tesis consiste en analizar la influencia de la actividad solar en la detección de rayos cósmicos galácticos y determinar cómo se puede extraer información pertinente sobre el ciclo solar y el clima espacial a partir de los datos del Observatorio. Para llevar a cabo este propósito, se realizó un análisis exhaustivo de los scaler, desde el año 2006 hasta el 2021 gran parte correspondiente al ciclo solar 24 y se contrastaron con datos provenientes de otros experimentos y fuentes, tales como monitores de neutrones, y satélites en el caso de los datos de manchas solares y el viento solar. El estudio se enfocó en examinar la correlación entre el flujo de rayos cósmicos y los parámetros solares, así como la respuesta del Observatorio a eventos solares transitorios, como erupciones solares y eyecciones de masa coronal. Los resultados revelan que el Observatorio Pierre Auger exhibe sensibilidad a las variaciones del clima espacial tanto a corto como a largo plazo, y puede utilizarse como un instrumento complementario valioso para la investigación de la actividad solar y sus efectos sobre la Tierra.
}\normalsize
\clearpage