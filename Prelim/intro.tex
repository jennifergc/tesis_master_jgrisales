\addcontentsline{toc}{chapter}{Introducción}
\newpage
\chapter*{Introducción}
\label{sec:intro}

\noindent El Sol es la estrella más cercana a la Tierra y la principal fuente de energía para la vida en nuestro planeta, cuya actividad se manifiesta en forma de cambios en su luminosidad, campo magnético, viento solar y transporte de partículas además que presenta diferentes escalas temporales, desde minutos hasta siglos. La más conocida es su periodo de 11 años que se caracteriza por el aumento y la disminución del número de manchas solares donde se concentra el campo magnético debido al mecanismo de convección del plasma (\cite{Balogh_2008}).

El estudio exhaustivo de la dinámica solar y todos los fenómenos determinados por la interacción entre el viento solar, el campo magnético interplanetario y el campo magnético terrestre, se conoce hoy como \textit{Clima Espacial}. El término se empezó a usar en los años 50 y se popularizó en los años 90 (\cite{cade_2015}). Sin embargo antes de eso ya se habían observado y caracterizado algunos fenómenos de clima espacial como las auroras, las tormentas magnéticas o los rayos cósmicos. Por ejemplo, el primer registro de una tormenta magnética fue realizado por el naturalista Alexander von Humboldt, en 1808 relacionándola con las manchas solares, o el registro en 1859 de la mayor tormenta magnética conocida \textit{el evento Carrington}, que causó auroras espectaculares y daños en las líneas telegráficas (\cite{hayakawa_2019}). En el siglo XX, el desarrollo de la tecnología espacial hizo que el clima espacial fuera más relevante, ya que podía afectar a los satélites, las comunicaciones, la navegación o la salud de los astronautas (\cite{Schrijver_2015}) y se comenzaron a desarrollar instrumentos tanto en el espacio como en la Tierra, que permiten observar el Sol, el viento solar, el campo magnético interplanetario, el campo magnético terrestre, la ionosfera y la atmósfera. 

Entre estos instrumentos se encuentran los detectores de rayos cósmicos: partículas cargadas que provienen de diversas fuentes astrofísicas incluido el Sol, capaces de acelerar partículas a altas velocidades incluso cercanas a la de la luz abarcando un amplio espectro de energía.  Sin embargo es un hecho que al aumentar la energía, disminuye el número de partículas que pueden ser detectadas, haciendo al Sol y a la galaxia como la principal fuente y generando así lo que llamamos: fondo de rayos cósmicos (\cite{gaisser_2016}). Todas las observaciones realizadas hasta la fecha han comprobado que los diferentes fenómenos solares son la principal causa de alteración de este flujo de fondo creando un vínculo directo entre la medición de estas partículas y la actividad solar. 

Uno de los experimentos más importantes construidos para la medición indirecta de rayos cósmicos es el Observatorio Pierre Auger, que se encuentra en la provincia de Mendoza, Argentina, y tiene como objetivo estudiar partículas de ultra alta energía (UHECR), es decir, aquellas con energías superiores a los $10^{18}eV$. Sin embargo, a partir del 2005 este observatorio implementó un sistema de detección adicional de baja energía denominado \textit{modo scaler} que registra el flujo de fondo en forma de número de partículas por segundo, con el objetivo de observar destellos de rayos gamma y eventos transitorios como Forbush Decreases. Por lo tanto, este nuevo criterio de almacenamiento de eventos ha abierto un mundo de nuevos análisis e información física que en los últimos años ha logrado mayor interés en la comunidad científica.

La presente tesis de maestría, hace parte de este conjunto de trabajos que busca caracterizar al arreglo de detectores de superficie como un sensor de alta resolución del rayos cósmicos galácticos. Mas específicamente, aquí se presenta un estudio de los efectos de la actividad solar a largo plazo sobre el flujo de rayos cósmicos secundarios en el Observatorio Pierre Auger, utilizando el modo Scaler, considerando el efecto que tienen los eventos transitorios en estas mediciones. 

Para ello, se ha realizado un análisis de los datos del modo Scaler del Observatorio Pierre Auger, desde el año 2006 hasta el año 2021, y se han comparado con los datos de otros detectores y fuentes, como los monitores de neutrones, los satélites, las manchas solares y el campo magnético interplanetario. Se ha estudiado la correlación entre el flujo de rayos cósmicos y los parámetros solares, así como la respuesta del Observatorio a los eventos solares transitorios, como las erupciones solares y las eyecciones de masa coronal y su afectación al ciclo solar de 11 años.

%%%%%%%%%%%%
