%\addcontentsline{toc}{chapter}{Agradecimientos}
\newpage
\chapter*{Agradecimientos}
Dos cirugías, una mente quebrada, una mano rota y un corazón partido no fueron suficientes para apagar la pasión que hay en mí por la ciencia. Ningún proceso es fácil y el mío tampoco lo fue, pero todas las dificultades me permitieron entender lo que representa para mí la ciencia y el aprendizaje continuo. Y a pesar de lo que parece ser una tragedia, solo fue una de las caras del dado de la vida, que te quita, te pone, te sube y te baja. Hice muchas cosas en esta maestría, aprendí demasiado y disfruté cada momento donde pude nadar en los libros, artículos, nuevos descubrimientos y fenómenos físicos tanto inimaginables como aparentemente triviales. Además de eso enseñé. Fui docente de secundaria y pude volver a ser niña, montarme en los pupitres, hacer garabatos y experimentos locos esta vez como la anfitriona de la fiesta, guiando y transmitiendo ese amor a la física que nadie pone en duda cuando ve mis ojos y mi sonrisa.

Amo la física, amo la decisión que tomé de arriesgarme por una carrera que nadie conocía y que era inaccesible para mí, con los pocos recursos que contaba y con la obligación de marcharme de mi humilde hogar con la convicción romántica de acercarme cada vez más a las estrellas. Pero no lo habría logrado sola, sin la ayuda de la gente que si yo merecer absolutamente nada en esta vida, me ayudó, me ayuda y me impulsa, simplemente porque sí, a cumplir mi sueño. Son demasiadas personas, muchas, desde mi profesor Luis Nuñez a quien admiro profundamente y es sin lugar a dudas mi principal modelo a seguir en la academia y en la vida, mis amigos, Luzma desde la distancia, Edwin, Fer, Los Jesuses de mi vida, Tatiana, la selección de rugby UIS, hasta Elba y Wilson mis padres, que con el tiempo que lo arregla todo, comprendieron también mi pasión.

Ahora entrego un documento que habla del Sol, que fascinante y maravilloso que es, escribir