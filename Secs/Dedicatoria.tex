% \addcontentsline{toc}{chapter}{Dedicatoria}
  
%   \cleardoublepage
\newpage

\vspace*{\stretch{3}}

\begin{flushleft}
\textit{¿En perseguirme, mundo, qué interesas?}\newline
\textit{¿En qué te ofendo, cuando sólo intento}\newline
\textit{poner bellezas en mi entendimiento}\newline
\textit{y no mi entendimiento en las bellezas?}\newline
\textit{Yo no estimo tesoros ni riquezas,}\newline
\textit{y así, siempre me causa más contento}\newline
\textit{poner riquezas en mi entendimiento}\newline
\textit{que no mi entendimiento en las riquezas.}\newline
\textit{Y no estimo hermosura que vencida}\newline
\textit{es despojo civil de las edades}\newline
\textit{ni riqueza me agrada fementida,}\newline
\textit{teniendo por mejor en mis verdades}\newline
\textit{consumir vanidades de la vida}\newline
\textit{que consumir la vida en vanidades.}\newline
\textit{``Quéjase de la suerte" por Juana Inés de Asbaje.}\newline

\textit{Este trabajo va dedicado a todas las mujeres que han defendido nuestro derecho al conocimiento.}
\end{flushleft}



\clearpage