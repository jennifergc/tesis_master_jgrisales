\newpage


\begin{itemize}
\item \textbf{Objetivo General}

Caracterizar cuantitativamente el efecto de los diferentes perfiles atmosféricos sobre el flujo de partículas secundarias originados por la interacción de rayos cósmicos, de baja energía, con la atmosféra.

\item \textbf{Objetivos Espec\'ificos}
\end{itemize}
\begin{itemize}
\item Generar perfiles atmosféricos de los 12 meses del a\~no a partir de datos del GDAS para ser usados en las simulaciones del fondo de partículas secundarias generadas por rayos cósmicos.
\item Cuantificar las diferencias en el flujo de secundarios entre la atmósfera predeterminada en CORSIKA y los perfiles construidos con GDAS para Bucaramanga.
\item Estimar el cambio del flujo de secundarios dependiendo de la época del año.
\item Desarrollar una metodología para la creación de perfiles atmosféricos con GDAS que se usará en las simulaciones
de fondo de rayos cósmicos de todos los sitios LAGO.
\end{itemize}

\newpage
