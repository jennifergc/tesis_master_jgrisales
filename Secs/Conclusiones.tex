\newpage
\chapter{Conclusiones}

Los datos de Scaler del observatorio pierre auger tienen una alta resolución y sensibilidad, lo que le permite identificar eventos a corto plazo. Se detectó una periodicidad diaria que está fuertemente asociada a la rotación de la tierra.

Se observó que el comportamiento de los scaler a lo largo del tiempo coincide con el de las estaciones de neutrones entre más cercanía haya con su rigidez de corte geomagnético.

Se realizó un estudio de periodicidad en donde se aplicó el método de Blackman-Tukey para estimar la densidad espectral de potencia mediante la ponderación de una función de auto correlación. Esto permitió observar periodicidades de 1 día, 30 y 365 días.

Se realiza un análisis de correlación para cuantificar la posible modulación de la actividad solar con el flujo de rayos cósmicos galáctico medido. Los parámetros usados fueron el número de manchas solares y el viento solar, observando correlaciones más fuertes con el número de manchas solares. 

Estos resultados son consistentes con lo observado para estaciones de neutrones altamente sensibles al flujo de GCR como Oulu. Observando también coeficientes derca del 0.6 para el periodo de 11 años.

Se realizó un filtrado de la señal de scaler en el tiempo, aplicando una ventana móvil de un día, con el objetivo de identificar eventos forbush y determinar la sensibilidad del observatorio a estos fenómenos.

Para el estudio de Forbush se usa la base de datos de IZMIRAN en donde se seleccionaron solo aquellos eventos de los que se haya confirmado la relación con eventos transitorios en la actividad solar. Este análisis permitió observar que el observatorio es sensible a Forbush con magnitud mayor o igual a 2, y estos corresponden con casi el $40\%$ de todos los eventos de ese tipo reportados en el tiempo de funcionamiento del observatorio y casi el $30\%$ de los eventos se presentan en la zona de mayor actividad solar. 

