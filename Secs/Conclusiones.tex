\newpage
\chapter{Conclusiones}


\noindent Se construyó una metodología que permite crear perfiles atmosféricos promediados mes a mes, para cualquier ubicación geográfica, usando el código GDASTOOL. Esta metodología propone la extracción de datos de dos horas del día diferentes: 0:00 y las 12:00 UTC-5, para todos los días del año. Se crearon perfiles atmosféricos para la ciudad de Bucaramanga (7.11 N, 73.11 E) y se construyeron 12 perfiles para cada mes del año 2018, que se contrastaron con el perfil atmosférico Subtropical que viene predeterminado en CORSIKA. Este trabajo, completa la secuencia de simulaciones que LAGO estableció en el programa de clima espacial, el cual busca estudiar los fenómenos relacionados a la  modulación que realiza el viento solar al flujo de secundarios que se pueden detectar en tierra.\\

Adicionalmente, se realizó una validación, construyendo perfiles atmosféricos para el Observatorio Pierre Auger, y contrastándolos con el perfil basado en GDAS que actualmente usa el Observatorio. Las simulaciones arrojaron que el modelo reconstruido reproduce el resultado de las atmósferas mensuales estándar de Malarg\"ue construidas con GDAS, con una diferencia porcentual por debajo del 20$\%$ después de los 400 $g/cm^{2}$. Además el comportamiento de la EAS que se obtiene con la atmósfera reconstruida muestra una diferencia de $\approx$ 2$\%$ en el valor del $X{max}$.\\

Con estos perfiles construidos, se contrastó la diferencia de  densidad atmosférica, en relación al perfil subtropical predeterminado en CORSIKA, y se observaron diferencias que llegan hasta el 58$\%$, en los primeros 30 km como se mostró en la figura \ref{fig:fig16}.\\
%Se observa una diferencia notable que se incrementa rápidamente en los primeros 4 km hasta llegar a los 250 $g/cm^{2}$. Esto corresponde a un 58$\%$ de diferencia. Luego esta diferencia  disminuye, pero se mantiene por encima del 40$\%$.

Así mismo, se estudió el efecto de estos perfiles atmosféricos en el flujo de fondo de secundarios sobre Bucaramanga y se confirmó la relación que tienen estos perfiles con la variación de temperatura mensual a lo largo del año. Se evidenció una diferencia en el flujo total entre 10,22$\%$ y el 24,12$\%$ correspondientes a los meses de noviembre y abril respectivamente. De forma similar, para los muones estas diferencias están entre 9,58$\%$ y 22,25$\%$. Este resultado permite confirmar que las variaciones atmosféricas a lo largo del año, pueden ser evidenciadas en el flujo de secundarios a nivel del suelo.\\

%una diferencia significativa, entre los perfiles mensuales y el subtropical. Es constante en el transcurso del año, permitiendo concluir que este perfil predeterminado por CORSIKA no es el perfil adecuado para la ciudad, puesto que sobrestima el flujo significativamente.\\
%%se obtiene una diferencia en el flujo total entre 10,22$\%$ y el 24,12$\%$ correspondientes a los meses de noviembre y abril respectivamente. De forma similar, para los muones estas diferencias están entre 9,58$\%$ y 22,25$\%$. Este resultado responde finalmente a la interrogante planteada en el capítulo anterior. Además, permite concluir que las variaciones atmosféricas a lo largo del año, pueden ser evidenciadas en el flujo de secundarios a nivel del suelo, incluso en regiones tropicales como Bucaramanga.\\
También, se estudió el efecto de los perfiles atmosféricos en el desarrollo de las EAS generadas por partículas individuales. Se encontró que el perfil atmosférico cambia levemente la distribución longitudinal de secundarios, observando un mayor número de partículas cerca al $X_{max}$. Sin embargo, las diferencias en la posición del $X_{max}$ no superan el 8$\%$.\\

Finalmente, cabe resaltar que la metodología presentada, puede ser usada en cualquier ubicación geográfica, donde se requieran perfiles atmosféricos en base a GDAS, que recojan información suficiente y estable de las variables atmosféricas, en periodos determinados de tiempo. Los códigos que permiten extraer y crear los perfiles atmosféricos mensuales están disponibles de forma libre, en el repositorio web: \url{https://github.com/jennifergc/Tesis_Pregrado}. \\

%Esta cadena de simulaciones se establece en tres bloques principales: el cálculo de los efectos del campo geomagnético en la propagación de partículas cargadas, que contribuyen a la radiación de fondo a nivel del suelo y que se caracterizan por la rigidez de corte en cada sitio LAGO \footcite{lagoSW}. La estimación del flujo de partículas secundarias al nivel de los detectores, originado por la interacción de los primarios con la atmósfera, parte en la que se concentra este trabajo. Y, la obtención de la respuesta del detector para los diferentes tipos de  partículas secundarias \footcite{andrei}.

Los resultados de este trabajo de grado, han sido presentados en  el VI Congreso Colombiano de Astronomía y Astrofísica CoCoA, celebrado en la ciudad de Medellín, con la ponencia: \textit{Estimación del flujo de astropartículas usando el Sistema Global de Asimilación de Datos, para la colaboración LAGO}, y en el 11th Workshop of the Latin American Giant Observatory LAGO, celebrado en la ciudad de Buenos Aires, con la ponencia: \textit{Generación de perfiles atmosféricos con GDAS para LAGO}.

