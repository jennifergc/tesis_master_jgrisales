\newpage
\chapter{Conclusiones}

Se realizó un análisis de la intensidad de rayos cósmicos CRI medido en el Observatorio Pierre Auger a través de las mediciones en modo \textit{scaler}. Para tal fin se tuvieron en cuenta los registros  que corresponden al periodo 2006-2021, cubriendo todo el ciclo solar 24.  Usando diferentes métodos para el análisis de series de tiempo, se observó que los \textit{scaler} tienen una respuesta a la modulación de la actividad solar a largo plazo sobre el flujo de GCR en concordancia a su rigidez de corte geomagnético $R_C=9,8 GV$ en comparación con otros detectores de neutrones ubicados a diferentes latitudes alrededor del mundo. 

La correlación entre la intensidad de rayos cósmicos y el histórico de número de manchas solares es moderada (entre  0.30 y 0.50) como se observó en la (figura \ref{fig:sunspots_corr}). Este resultado se encuentra en el mismo rango que el obtenido para la estación de Tsumeb que tiene una rigidez de corte de $R_C=9.15 GV$. Esta simulitud sugiere que el resultado puede estar relacionado a la rigidez de corte geomagnético que disminuye la sensibilidad considerablemente. Para el análisis de correlación, también se usó como valor de referencia la estación de Oulu cuyos datos han sido ampliamente estudiados por la comunidad científica debido a su rigidez de corte $R_C=0.81 GV$ , accesibilidad y consistencia en las mediciones a lo largo del tiempo. Para esta estación se observa una correlación fuerte mayor al 0.5  para el número de manchas solares.  Por otro lado, se observó una correlación débil (entre 0.10 y 0.29) con el viento solar esto se mantuvo también para las dos estaciones de neutrones consideradas en este estudio. La baja anticorrelación observada entre la intensidad de los rayos cósmicos y la velocidad del viento solar puede atribuirse a las propiedades del viento solar, específicamente el viento solar rápido, que se origina en los agujeros coronales del Sol y es notablemente estable (\cite{Oloketuyi_2020}).
%Especialmente hemos replicado los análisis de correlación efectuados en el trabajo de Oloketuyi y colaboradores \cite{Oloketuyi_2020}. 

También se realizó un análisis espectral a través de la transformada de Fourier y la función de autocorrelación que arroja una componente diaria, mensual y anual (figura\ref{FFT1}). Estas modulaciones han sido identificadas y analizadas previamente en la estaciones de neutrones y están relacionadas con la forma como el campo magnético de la Tierra interactúa con la heliósfera y sus perturbaciones asociadas la rotación solar. 

De la misma forma, se hizo un análisis cualitativo de la sensibilidad que tiene el observatorio a modulaciones a corto plazo, para identificar posibles efectos en la tasa de incidencia de eventos transitorios y la modulación en el ciclo solar. Para esto se utilizó el catálogo de eventos Forbush y disturbios interplanetarios del IZMIRAN (El Instituto Pushkov de Magnetismo Terrestre, Ionosfera y Propagación de Ondas de Radio de la Academia Rusa de Ciencias). De allí se seleccionaron aquellos eventos que estuvieran asociados a ondas de choque interplanetarias OCI o a una tormenta geomagnética de inicio repentino SSC (eventos asociados a ICME) (ver figura \ref{fig:FD_events}). Como resultado, se identificó una muestra de 148 eventos entre el 2006 y 2021 (con un nivel confianza razonable). Se observó que la mayor incidencia de FD $55.41\%$  ocurre en la zona de máxima actividad solar, específicamente en la zona de mayor anomalía en elcomportamiento de los Scaler medidos. En esta región también se encuentra el $71.20\%$ de los eventos con magnitud $>= 2\%$ respecto a todo el periodo de registro del Observatorio.

Ahora, disminuyendo cada vez más la ventana de observación de la señal, se realizó un filtrado de la modulación diaria sobre los datos de scaler y se seleccionó para su análisis el FE del 22 de junio del 2015 reportado en la base de datos con una magnitud del $9.6\%$ generada por dos llamaradas solares que corresponden al segundo evento más intenso del ciclo solar 24 y el mayor del que se tienen datos de scaler disponibles en el Observatorio Pierre Auger (figura \ref{22junio2015}). A partir de la señal original y la señal filtrada se obtuvo la densidad espectral de potencia que permitió ver con mayor nivel de detalle la zona de más alta frecuencia. Se observó evidencia de modulaciones a 24, 12, 8, 6 y 4.8 horas, reportadas también en algunos detectores de neutrones. La evidencia de la señal observada a 8 horas tiene muy pocos registros en la literatura y debe ser estudiada con mayor nivel de detalle. En general la explicación de las interacciones geomagnéticas detrás de estas modulaciones en la intensidad de rayos cósmicos, sigue siendo un problema abierto en este campo de estudio. 

Finalmente, se calculó la magnitud del FD obteniendo un porcentaje de decrecimiento del $3.01\%$ que es un valor muy por debajo de los que se reportan en la literatura (entre $5.2\%$ y el $15.2\%$) lo que sugiere que para determinar la capacidad de resolución de FD no recurrentes, se debe hacer una caracterización más robusta del fondo.