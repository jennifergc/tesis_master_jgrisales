%\addcontentsline{toc}{chapter}{Abstract}
\newpage
\chapter*{Abstract}
\label{sec:abst}
\footnotesize{
\noindent\textbf{TITLE:} CHARACTERIZATION OF ATMOSPHERIC PROFILES FOR THE LAGO SIMULATION CHAIN\astfootnote{Bachelor Thesis}

\noindent\textbf{AUTHOR:} Jennifer Grisales Casadiegos.\asttfootnote{Facultad de Ciencias. Escuela de Física. Director: Luis A. Núñez. Codirector: Christian Sarmiento.}

\noindent\textbf{KEYWORDS: } Astroparticle, particle flux, atmosphere, simulation, extensive air shower.

\noindent\textbf{DESCRIPTION:} One of the main objectives of the Space Weather program of the Latin American Giant Observatory (LAGO), is to study the influence of the solar activity on the secondary particle flux  variations, produced during the interaction between astroparticles with atmosphere. To this end, a chain of simulations is carried out, which estimates in detail, the Primary's development, from its entry into the Earth's atmosphere, to the Water Cherenkov Detector response. This work, complete the simulation chain, focusing their interest in to study the atmospheric effect on the secondary particle flux. To do this it developed a methodology that allows the creation and use of monthly atmospheric profiles, for any localization, in the EAS simulations within the CORSIKA code. Furthermore, the relevance to using the new monthly profiles it was checked, because they are able to reproduce in the secondary particle flux, the effect of the temperature changes along the year. This allows refine the estimates made.
}\normalsize
\clearpage