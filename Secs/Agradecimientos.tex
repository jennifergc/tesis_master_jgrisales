%\addcontentsline{toc}{chapter}{Agradecimientos}
\newpage
\chapter*{Agradecimientos}

Son muchas las personas que, sabiéndolo o no, fueron contribuciones significativas, en una combinación lineal que determinó el camino de mi formación como Física. A todos y todas, gracias. Fuertemente en mis recuerdos mis compañeros de aprendizaje Edwin Florez, Genderson Rueda y Yerson Barragán, con quienes pese a las dificultades, disfrutamos esta etapa universitaria formando no solo nuestro pensamiento científico, sino también construyendo una necesaria conciencia de clase que orientará nuestro quehacer académico y político por siempre. Debo agradecer a mi madre y mi padre, porque debido a su formación, desde mi infancia nació el sueño de vivir la ciencia. Personas como José Quintero que me alentaron a seguir adelante y sacar valentía de donde no había, gracias. También a Jesús Sánchez por abrirme las puertas de su casa, y su familia, gracias por todo el afecto y el apoyo incondicional. A mi mejor amiga Luz Marina Cabrera por ser mi confidente y por jugar conmigo a dibujarnos mutuamente alas para volar y soñar.

Gracias a Luis Núñez por inspirarme siempre y mostrarme que la ciencia es el camino. Gracias Christian Sarmiento por creer en mí y por su tiempo que valoré infinitamente. También gracias a Yeinzon Rodríguez porque me demostró que es posible seguir viendo la física con curiosidad y pasión.

Finalmente quiero agradecer a Jesús Peña haber sido mi apoyo emocional en todo este proceso. Por ser mi ejemplo e inspiración de esfuerzo, perseverancia, trabajo duro y amor genuino por la ciencia. Gracias por tanto amor.